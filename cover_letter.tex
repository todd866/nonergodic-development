\documentclass[11pt]{letter}
\usepackage[margin=1in]{geometry}
\usepackage{hyperref}
\usepackage{parskip}

\signature{Ian Todd\\Sydney Medical School\\University of Sydney\\itod2305@uni.sydney.edu.au}
\address{Ian Todd\\Sydney Medical School\\University of Sydney\\Sydney, NSW, Australia}

\begin{document}

\begin{letter}{Professor Abir Igamberdiev\\Editor-in-Chief\\BioSystems}

\opening{Dear Professor Igamberdiev,}

I am pleased to submit ``Nonergodic Development: How High-Dimensional Systems with Low-Dimensional Anchors Generate Phenotypes'' for consideration in \emph{BioSystems}.

This manuscript develops a theoretical framework for understanding why genotype does not algorithmically determine phenotype. The core insight is that biological development is \textbf{nonergodic}: the state space of a developing organism is too vast to explore in biological time. Developmental trajectories are trapped in attractor basins, and the phenotype reflects which basin was entered---not an algorithmic genotype-to-phenotype mapping.

The genome functions as a low-dimensional \textit{anchor} that constrains which attractor basins are accessible, while environmental history determines which basin the developmental trajectory enters. I formalize this via the \textbf{Dimensional Gap} ($\Delta_D$), which quantifies the mismatch between anchor dimensionality and developmental degrees of freedom. When $\Delta_D \gg 0$, allele-based and trajectory-based models become non-identifiable from aggregate data.

The ``Twin Worlds'' experiment demonstrates this dramatically: identical genotype distributions in different environmental regimes produce patterns that naive genetic analysis would misinterpret as allele frequency differences ($F_{ST} \approx 0$ yet $P_{ST} \gg 0$). This provides a formal explanation for ``missing heritability'' in GWAS---the heritability is not missing variants but missing trajectory information.

I apply this framework to Sierra et al.\ (2025, \emph{Science Advances}), who found that cooperative mammalian species have lower cancer prevalence. Their allele-based interpretation is compelling, but my framework shows it is not uniquely supported: cooperative environments may guide developmental trajectories into high-repair attractor basins that already exist in the genetically-defined landscape.

This work provides formal foundations for Lissek's (2024, \emph{BioSystems}) ``cancer memory'' hypothesis: in nonergodic terms, epigenetic memory \textit{is} the trapped state---the system cannot spontaneously return to its previous attractor basin.

This submission continues a research program on dimensional constraints in biology:
\begin{itemize}
    \item Todd (2025a): ``The limits of falsifiability'' (DOI: 10.1016/j.biosystems.2025.105608)
    \item Todd (2025b): ``Timing inaccessibility and the projection bound'' (DOI: 10.1016/j.biosystems.2025.105632)
\end{itemize}

The manuscript engages with recent \emph{BioSystems} publications including Lee et al.\ (2022) on plasticity, Letsou (2024) on temporal development, Corning (2022) on systems evolution, and Fontana (2023) on development-ageing-cancer connections. All simulation code is available at \url{https://github.com/todd866/nonergodic-development}.

This manuscript is original and not under consideration for publication elsewhere.

\closing{Thank you for your consideration.}

\end{letter}
\end{document}
