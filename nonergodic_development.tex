\documentclass[12pt]{article}

\usepackage[utf8]{inputenc}
\usepackage[T1]{fontenc}
\usepackage{lmodern}
\usepackage[margin=1in]{geometry}
\usepackage{amsmath,amssymb,amsthm}
\usepackage{graphicx}
\usepackage[numbers,square]{natbib}
\usepackage{hyperref}
\usepackage{booktabs}
\usepackage{setspace}
\doublespacing

% Theorems
\newtheorem{theorem}{Theorem}
\newtheorem{proposition}[theorem]{Proposition}
\newtheorem{lemma}[theorem]{Lemma}
\newtheorem{corollary}[theorem]{Corollary}
\theoremstyle{definition}
\newtheorem{definition}[theorem]{Definition}
\newtheorem{example}[theorem]{Example}
\theoremstyle{remark}
\newtheorem{remark}[theorem]{Remark}

% Shortcuts
\newcommand{\R}{\mathbb{R}}
\newcommand{\E}{\mathbb{E}}

\title{Nonergodic Development: \\
How High-Dimensional Systems with Low-Dimensional Anchors \\
Generate Phenotypes}

\author{Ian Todd\\
Sydney Medical School\\
University of Sydney\\
\texttt{itod2305@uni.sydney.edu.au}}

\date{}

\begin{document}

\maketitle

\begin{abstract}
Biological development is a high-dimensional dynamical process that cannot explore its state space in finite time---it is \textit{nonergodic}. We argue that this nonergodicity, combined with low-dimensional genetic anchors, is the fundamental reason why genotype does not algorithmically determine phenotype. The genome constrains which regions of developmental state space are reachable, but environmental history determines which attractor basin the system occupies. Using a minimal developmental network model, we demonstrate that (1) identical genotypes produce substantially different phenotypes depending on which trajectory the system follows, (2) these trapped states constitute ``developmental memory'' that is invisible to genetic analysis, and (3) the ``dimensional gap'' $\Delta_D$ between genetic parameters and developmental degrees of freedom quantifies this non-identifiability. We apply this framework to recent findings on cooperative lifestyles and cancer prevalence, showing that the pattern ``cooperative species have less cancer'' admits both allele-based and trajectory-based interpretations. The models diverge only under environmental intervention. Allele stories are not wrong but are projections that discard the nonergodic structure of development.
\end{abstract}

\textbf{Keywords:} nonergodicity, developmental dynamics, genotype-phenotype map, epigenetic memory, attractor basins, dimensional constraints

\section{Introduction}

Biological development unfolds in a high-dimensional state space. Gene regulatory networks, signaling cascades, and metabolic pathways create a system with vastly more degrees of freedom than the genome that parameterizes it. This dimensionality has a fundamental consequence: the system is \textit{nonergodic}---it cannot explore its state space in biological time. A developing organism follows one trajectory through developmental state space and ends up in one attractor basin, leaving the vast majority of possible states forever unvisited.

This nonergodicity is not a limitation to be overcome but a feature that enables stable phenotypes. The genome acts as a low-dimensional \textit{anchor} that constrains which regions of state space are reachable, while environmental history determines which specific attractor the system occupies within those constraints. The phenotype is not algorithmically determined by the genotype; it is the trapped state of a nonergodic system whose trajectory was shaped by both genetic and environmental inputs.

This framework resolves a persistent puzzle in genetics: why do genome-wide association studies fail to account for the majority of heritable variation in complex traits \cite{manolio2009,zuk2012}? The ``missing heritability'' is not missing genetic variants but rather the missing environmental history---the trajectory information that determined which attractor basin each individual occupies. Additive genetic models project a high-dimensional, trajectory-dependent reality onto a low-dimensional genotype space, discarding precisely the information that distinguishes individuals with identical genotypes.

We formalize this via the \textbf{Dimensional Gap} ($\Delta_D$), which quantifies the mismatch between genetic parameters and developmental degrees of freedom. When $\Delta_D \gg 0$, the mapping from genotype to phenotype is many-to-one: the same genotype can produce different phenotypes depending on which trajectory the system followed. This creates a fundamental \textit{non-identifiability} between allele-based and trajectory-based explanations for phenotypic patterns.

\subsection*{Application: Cooperative Lifestyles and Cancer}

We apply this framework to a recent finding in cancer evolution. Sierra et al.\ (2025) demonstrated that cooperative mammalian species exhibit lower cancer prevalence than competitive species, and modeled this as selection on ``oncogenic variants'' \cite{sierra2025}. Their allele-based interpretation is compelling---but our framework shows it is not uniquely supported by the data.

The same pattern admits a trajectory-based interpretation:
\begin{itemize}
    \item \textbf{Allele interpretation:} Cooperative lineages have accumulated cancer-suppressing alleles through selection.
    \item \textbf{Trajectory interpretation:} Cooperative environmental cues guide development into attractor basins with higher somatic repair allocation, yielding lower cancer mortality as an emergent property.
\end{itemize}

These interpretations are \textit{non-identifiable} from cross-species comparative data. They diverge only in predictions for intervention: the allele model predicts that changing an organism's environment will not change its cancer risk; the trajectory model predicts substantial phenotypic shifts.

This connects to Lissek's \cite{lissek2024} proposal of ``cancer memory''---epigenetic mechanisms that maintain malignant phenotypes once initiated. Our framework provides the information-theoretic foundation: the developmental system's nonergodicity \textit{is} the memory. Environmental perturbations shift the system to new attractor basins that the genome-as-anchor helps maintain.

In this paper, we:
\begin{enumerate}
    \item Develop the theoretical framework of nonergodic development with low-dimensional anchors (Section~\ref{sec:theory}).
    \item Formalize a minimal developmental network model that demonstrates these principles (Section~\ref{sec:model}).
    \item Prove a non-identifiability proposition showing that allele-based patterns can always be matched by trajectory-based mechanisms (Section~\ref{sec:proposition}).
    \item Demonstrate the framework with simulations, including a ``Twin Worlds'' experiment (Section~\ref{sec:results}).
    \item Discuss implications for evolutionary biology, GWAS, and intervention design (Section~\ref{sec:discussion}).
\end{enumerate}


\section{Theoretical Framework}\label{sec:theory}

\subsection{Nonergodicity in High-Dimensional Systems}

A dynamical system is \textit{ergodic} if, given sufficient time, its trajectory will visit every accessible region of state space with frequency proportional to the equilibrium measure. Ergodicity underlies much of statistical mechanics: it allows us to equate time averages with ensemble averages.

Biological development is not ergodic. The state space of a developing organism---the configuration of gene expression, protein concentrations, cellular positions, and tissue architectures---is astronomically large. A gene regulatory network with $N$ genes, each with $k$ expression levels, has $k^N$ possible states. For realistic values ($N \sim 20,000$, $k \sim 10$), this exceeds any timescale accessible to biological processes.

\begin{definition}[Developmental Nonergodicity]
A developmental system is \textbf{nonergodic} if its trajectory through state space cannot visit a representative sample of accessible states within the organism's lifetime. Formally, let $\mathcal{M}$ be the accessible manifold of developmental states. The system is nonergodic if:
\begin{equation}
    \frac{\text{Vol}(\text{trajectory})}{\text{Vol}(\mathcal{M})} \to 0
\end{equation}
as the dimensionality of $\mathcal{M}$ increases.
\end{definition}

This nonergodicity has a crucial consequence: the developmental trajectory is \textit{trapped}. Once the system enters an attractor basin, it cannot spontaneously explore alternative basins. The final phenotype reflects not the equilibrium distribution over possible states, but the particular basin the trajectory happened to enter.

\subsection{Beyond the Computational Metaphor}

This continuous dynamical systems perspective differs fundamentally from the dominant metaphor in molecular biology. Since the discovery of the genetic code, biological systems have been described using computational language: genes ``encode'' proteins, regulatory networks ``process'' signals, development follows a ``program.'' This framing---treating organisms as discrete-state machines executing algorithms---has been so successful and so pervasive that it is rarely questioned. Network diagrams with nodes and edges are the standard visual vocabulary; information flow through discrete processing steps is the default conceptual framework.

Yet the graph model carries a hidden puzzle: how do discrete signals get routed precisely through the molecular chaos of a living cell? The network diagram shows an edge from node A to node B, but physically there is no edge---only molecules diffusing in crowded cytoplasm.

We propose a complementary perspective grounded in continuous physics. Instead of discrete nodes and edges, we have continuous state spaces and smooth trajectories. Instead of algorithms that map inputs to outputs, we have dynamical systems that \textit{flow} through high-dimensional landscapes. The genome does not ``compute'' the phenotype---it shapes the landscape through which the developmental trajectory flows. The routing puzzle dissolves: there are no discrete signals being routed, only continuous trajectories constrained by the physics of the substrate.

This shift matters because computational metaphors implicitly assume something like ergodicity: that given enough time or trials, the system can explore its state space. Continuous high-dimensional dynamics reveals why this assumption fails in biological time. The state space is too vast; trajectories are trapped. The phenotype is not the output of a computation but the basin into which a physical system settled.

\subsection{Low-Dimensional Anchors}

If developmental state space is vast and trajectories are trapped, what prevents phenotypic chaos? The answer is \textit{anchoring} by low-dimensional constraints.

\begin{definition}[Developmental Anchor]
An \textbf{anchor} is a low-dimensional structure that constrains which regions of developmental state space are reachable. The genome is the primary anchor, specifying regulatory network topology, protein sequences, and signaling pathways. The epigenome provides secondary anchoring, modulating which regions of the genome-defined landscape are accessible at any given time.
\end{definition}

Together, genetic and epigenetic anchors define the ``landscape'' of possible developmental trajectories. They do not specify which attractor basin the system will occupy---that depends on the trajectory, which depends on environmental history. But the anchors constrain \textit{which basins exist} and \textit{which are reachable} from a given starting point.

This is precisely Waddington's \cite{waddington1957} epigenetic landscape, reframed in dynamical systems terms: the genome shapes the hills and valleys; the developmental trajectory is a ball rolling down one particular path. Different environmental perturbations can push the ball into different valleys---but not into valleys that the landscape doesn't contain.

\subsection{The Dimensional Gap}

We quantify the relationship between anchor dimensionality and state space dimensionality:

\begin{definition}[Dimensional Gap]
Let $L$ be the dimension of the genotype (anchor) space and $k$ the dimension of measured phenotypic traits. Let the developmental system evolve on a manifold $\mathcal{M}$ with \textbf{effective dimension} $m_{\text{eff}}$. The \textbf{dimensional gap} is:
\begin{equation}
    \Delta_D = m_{\text{eff}} - (L + k)
\end{equation}
\end{definition}

When $\Delta_D \gg 0$, the developmental system has far more degrees of freedom than can be specified by the genome or captured by phenotypic measurement. This creates a fundamental ambiguity:
\begin{itemize}
    \item The same genotype can produce different phenotypes (depending on which trajectory/attractor)
    \item The same phenotype can arise from different mechanisms (genotype-determined vs.\ trajectory-determined)
\end{itemize}

The dimensional gap is the source of non-identifiability. Projecting from the high-dimensional developmental reality to the low-dimensional genotype-phenotype pair destroys the trajectory information that distinguishes otherwise identical outcomes \cite{todd2025falsifiability,todd2025maxwell}.

\subsection{Epigenetic Memory as Trapped States}

The connection to epigenetic memory is now clear. ``Epigenetic memory'' refers to heritable cellular states that are not encoded in the DNA sequence---persistent patterns of gene expression, chromatin modification, or cellular identity that are maintained across cell divisions.

In our framework, epigenetic memory is simply a \textit{trapped state} in a nonergodic developmental system. The genome (anchor) constrains which states are stable, while environmental history determines which stable state the system entered. Once entered, the state persists because the system cannot spontaneously explore alternatives.

This provides the information-theoretic foundation for Lissek's \cite{lissek2024} ``cancer memory'' hypothesis: oncogenic changes may push the developmental system into a new attractor basin, where the nonergodic dynamics maintain the malignant phenotype without requiring continued mutation accumulation.

\subsection{Application Context: Cancer and Cooperation}

Sierra et al.\ (2025) found that cooperative mammalian species have lower cancer prevalence than competitive species \cite{sierra2025}. Their model explains this via allele frequency differences: selection favors cancer-suppressing alleles in cooperative lineages where older individuals contribute as helpers.

Our framework suggests an alternative: cooperative environmental cues may guide developmental trajectories into attractor basins with higher somatic repair allocation. The genome constrains which basins are available; the social environment determines which basin is entered. Species that consistently experience cooperative environments would consistently develop into ``low-cancer'' attractor basins---producing the same aggregate pattern as allele-based selection, but via a different mechanism.

More generally, any species experiencing variable environments across generations would be expected to evolve genomes that anchor multiple attractor basins---allowing environment-dependent phenotype determination rather than fixed strategies. Singh and Glowacki \cite{singh2022} documented such environmental variability in human evolutionary history, but the principle applies broadly: when environments vary, selection favors trajectory-responsive developmental systems over rigidly canalized ones.


\section{Model}\label{sec:model}

We construct a minimal developmental network that instantiates the theoretical framework: the genotype serves as a low-dimensional anchor, the developmental state evolves in a high-dimensional space, and environmental history shapes the trajectory through that space.

\subsection{The Developmental System}

Let $g \in \R^L$ be the genotype (the \textit{anchor}, with $L$ small), $e_t \in \R^p$ be the environmental input at time $t$, and $h_t \in \R^m$ be the developmental state (with $m \gg L$).

The developmental dynamics follow a recurrent network:
\begin{equation}\label{eq:dynamics}
    h_{t+1} = \sigma\left( W_h h_t + W_e e_t + W_g g \right)
\end{equation}
where $W_h, W_e, W_g$ are weight matrices and $\sigma = \tanh$ provides nonlinearity.

The key features of this model:
\begin{itemize}
    \item \textbf{Nonergodicity:} The state space $\R^m$ is high-dimensional; any single trajectory visits a negligible fraction of it.
    \item \textbf{Anchoring:} The genotype $g$ enters at every timestep, constraining which regions of state space are accessible.
    \item \textbf{Trajectory-dependence:} The environmental history $\{e_0, e_1, \ldots, e_T\}$ determines which specific trajectory is followed within the anchor's constraints.
    \item \textbf{Attractor dynamics:} The recurrent structure creates attractor basins; different trajectories can converge to different attractors.
\end{itemize}

\subsection{Phenotype as Trapped State}

The phenotype $x \in \R^n$ is a readout of the final developmental state:
\begin{equation}
    x = W_{out} h_T
\end{equation}

This final state $h_T$ is a \textit{trapped state}---the endpoint of a nonergodic trajectory. The same genotype $g$ can produce different $h_T$ (and thus different $x$) depending on which trajectory was followed.

We extract a ``repair allocation'' variable $r \in [0,1]$ from the phenotype, representing the organism's investment in somatic maintenance versus reproduction---a classic life-history tradeoff \cite{stearns1992,roff2002}. The emergent cancer mortality is:
\begin{equation}\label{eq:mu}
    \mu_S(x) = \mu_0 (1 - \alpha \cdot r(x))
\end{equation}
where $\mu_0$ is baseline mortality and $\alpha \in (0,1)$ is the efficacy of repair.

\subsection{Environmental Regimes}

We define two environmental regimes that guide trajectories into different attractor basins:
\begin{itemize}
    \item \textbf{Cooperative:} Low variance, high social support, low conflict $\to$ trajectories toward high-repair attractors.
    \item \textbf{Competitive:} High variance, low support, high conflict $\to$ trajectories toward low-repair attractors.
\end{itemize}

\noindent\textbf{Parameter values:} We use $L = 5$ (anchor dimension), $m = 20$ (developmental state dimension), $n = 10$ (phenotype dimension), and $p = 3$ (environmental input dimension). This gives $\Delta_D = m - (L + 1) = 14$ when measuring a single trait ($\mu_S$). Cancer parameters: $\mu_0 = 0.1$, $\alpha = 0.8$. See code for implementation details.

\subsection{Contrasting Interpretations}

Figure~\ref{fig:dags} illustrates how the same outcome admits two causal interpretations:
\begin{itemize}
    \item \textbf{Allele model:} $G \to \mu_S \to \text{Cancer}$. The genotype algorithmically determines mortality; environment only modulates selection pressure over evolutionary time.
    \item \textbf{Trajectory model:} $G \to \text{Dev} \to \pi(E) \to r_t \to \mu_S$. The genotype anchors a developmental system; environment shapes the trajectory; the trapped state determines mortality.
\end{itemize}
Both produce the same aggregate correlation (Environment $\leftrightarrow$ $\mu_S$) when projected to observables---this is the non-identifiability created by $\Delta_D > 0$.


\section{Non-Identifiability Result}\label{sec:proposition}

\begin{proposition}[Non-identifiability of allele vs.\ policy mechanisms]\label{prop1}
Let Model A be any allele-based system where late-life mortality takes values $\{\mu, \mu + \delta\}$ across two ecological regimes, with $0 \leq \mu < \mu + \delta \leq \mu_0$. Let Model B be any plasticity-based developmental system \textbf{capable of realizing any repair allocation $r \in [0,1]$ via some environmental history}, and where $\mu_S = \mu_0(1 - \alpha r)$ as in Equation~\ref{eq:mu}. Then for any pair $(\mu, \mu + \delta)$ satisfying these constraints, there exists an environment pair $(E_{\text{coop}}, E_{\text{comp}})$ such that the induced $\mu_S$ distributions in Model B match those in Model A for all species-level summary statistics.
\end{proposition}

\begin{proof}
Take any $(\mu, \mu + \delta)$ from Model A. We construct environments that induce matching $\mu_S$ distributions in Model B.

Choose $r_{\text{coop}}$ and $r_{\text{comp}}$ such that:
\begin{align}
    \mu_0(1 - \alpha r_{\text{coop}}) &= \mu \\
    \mu_0(1 - \alpha r_{\text{comp}}) &= \mu + \delta
\end{align}

Solving: $r_{\text{coop}} = (1 - \mu/\mu_0)/\alpha$ and $r_{\text{comp}} = (1 - (\mu + \delta)/\mu_0)/\alpha$.

For sufficiently expressive dynamics (e.g., the recurrent network of Equation~\ref{eq:dynamics}), there exists a mapping from environmental histories to any desired $r$; this is empirically illustrated in Figure~\ref{fig:same_genotype}D. Thus there exist environments $E_{\text{coop}}$ and $E_{\text{comp}}$ that induce these repair levels.

Therefore, any species-level summary (mean $\mu_S$, variance, etc.) that Model A produces can be exactly matched by Model B with appropriate environment choice. The models are observationally equivalent at this level.
\end{proof}

\begin{remark}
This result formalizes the sense in which allele models are ``incomplete'': they are not wrong, but they are \textit{non-identifiable} from aggregate data. The models become distinguishable only through interventions that change environment while holding genotype fixed.
\end{remark}


\section{Results}\label{sec:results}

All figures show simulated outputs from the developmental network model; no empirical data are used. The simulations demonstrate the core theoretical claims: trajectory-dependence, attractor dynamics, and non-identifiability.

\subsection{Trajectory Divergence: Same Anchor, Different Attractors}

Figure~\ref{fig:same_genotype} demonstrates the fundamental prediction of nonergodic development: identical genotypes (anchors) produce different phenotypes when developmental trajectories follow different paths through state space.

\begin{figure}[htbp]
    \centering
    \includegraphics[width=\textwidth]{figures/fig1_same_genotype.pdf}
    \caption{\textbf{Trajectory divergence under identical anchoring.} (A) Two environmental histories: cooperative (blue) and competitive (red). (B) Developmental trajectories through state space (PCA projection); same starting point (green circle), different endpoints (squares). (C) Different trapped states produce different phenotypes. (D) Different attractor basins yield different repair allocations and cancer mortality---same genotype, different outcomes.}
    \label{fig:same_genotype}
\end{figure}

Panel (B) is the key visualization: the two trajectories diverge in state space despite sharing the same anchor constraints. The endpoints represent different attractor basins---trapped states that persist because the system cannot spontaneously explore alternatives. Panel (D) shows the phenotypic consequence: different repair allocations and thus different cancer mortality from identical genotypes.

\subsection{Population-Level Non-Identifiability}

Figure~\ref{fig:population} shows that aggregate population patterns cannot distinguish between allele-based and trajectory-based mechanisms.

\begin{figure}[htbp]
    \centering
    \includegraphics[width=\textwidth]{figures/fig2_population_patterns.pdf}
    \caption{Non-identifiability at the population level. (A) Trajectory model: same genetic distribution in different environments $\to$ different $\mu_S$ distributions. (B) Allele model: different genetic distributions can produce matching patterns. (C) Aggregate statistics are indistinguishable---this is the $\Delta_D$ projection problem.}
    \label{fig:population}
\end{figure}

\subsection{Causal Structure}

Figure~\ref{fig:dags} contrasts the causal graphs. In the allele model, the genotype algorithmically determines the phenotype; environment acts only through evolutionary selection. In the trajectory model, the genotype anchors a developmental system that integrates environmental inputs in real time.

\begin{figure}[htbp]
    \centering
    \includegraphics[width=\textwidth]{figures/fig3_causal_dags.pdf}
    \caption{Contrasting causal structures. (A) Allele model: $G \to \mu_S$, environment external. (B) Trajectory model: $G$ anchors development, $E$ shapes trajectory, $\mu_S$ emerges from trapped state.}
    \label{fig:dags}
\end{figure}

\subsection{The Dimensional Gap in Action}

Figure~\ref{fig:projection} visualizes the information loss that occurs when projecting from the high-dimensional (genotype $\times$ environment $\times$ trajectory) space to the low-dimensional (genotype $\to$ phenotype) summary.

\begin{figure}[htbp]
    \centering
    \includegraphics[width=\textwidth]{figures/fig4_projection_loss.pdf}
    \caption{Projection destroys trajectory information. (A) Full space: environments clearly separate. (B) Projected space: trajectory information lost. (C) Variance decomposition: environment explains most $\mu_S$ variance, but allele models misattribute this to ``genetic effects''---the dimensional gap in action.}
    \label{fig:projection}
\end{figure}

\subsection{Twin Worlds: The Decisive Experiment}

Figure~\ref{fig:twins} provides the clearest demonstration of nonergodic development. We create two ``worlds'' with \textit{identical} genotype distributions but different environmental regimes. The developmental trajectories in each world converge to different attractor basins, producing dramatically different phenotype distributions.

\begin{figure}[htbp]
    \centering
    \includegraphics[width=\textwidth]{figures/fig5_twin_worlds.pdf}
    \caption{Twin worlds experiment. (A) Identical genomes, different worlds $\to$ different $\mu_S$ distributions. (B) A naive allele analysis would infer ``oncogenic allele'' frequency differences where none exist. (C) Genotype (anchor) shows minimal correlation with $\mu_S$; trajectory determines outcome.}
    \label{fig:twins}
\end{figure}

The genetic distance between worlds is $F_{ST} \approx 0$ (by construction), yet phenotypic distance is $P_{ST} \gg 0$. A GWAS would find no significant variants and conclude ``missing heritability''---but the heritability is not missing, it is \textit{trajectory-based}. The anchor constrains which attractors exist; the environment determines which attractor is entered.

\subsection{Divergent Predictions: The Intervention Test}

Figure~\ref{fig:prediction} shows where the models diverge. The allele model predicts that phenotype is locked by genotype---environmental change cannot alter the trapped state. The trajectory model predicts that environmental change can guide the system toward a new attractor.

\begin{figure}[htbp]
    \centering
    \includegraphics[width=\textwidth]{figures/fig6_intervention_test.pdf}
    \caption{The intervention test. (A) Allele model: no change when environment shifts (genotype determines phenotype). Trajectory model: substantial shift (new environment $\to$ new attractor). (B) Individual trajectories converge toward the new attractor---phenotypic change without genetic change.}
    \label{fig:prediction}
\end{figure}

This is the critical empirical test. If moving organisms from competitive to cooperative environments produces phenotypic shifts toward higher repair allocation, the trajectory model is supported. If phenotypes remain stable, the allele model is supported.


\section{Discussion}\label{sec:discussion}

\subsection{Nonergodicity as the Core Insight}

The central contribution of this paper is recognizing that developmental systems are \textit{nonergodic}. The state space is too vast to explore; trajectories are trapped; phenotypes reflect which attractor was entered, not an algorithmic genotype-to-phenotype mapping.

This reframes several longstanding puzzles:
\begin{itemize}
    \item \textbf{Missing heritability:} Not missing variants, but missing trajectory information. GWAS project out the environmental history that determines which attractor each individual occupies.
    \item \textbf{Epigenetic memory:} Not a separate mechanism, but trapped states in nonergodic dynamics. The genome anchors; the trajectory traps.
    \item \textbf{Genetic assimilation:} How plastic responses become canalized. If an attractor basin is consistently entered across generations, selection can modify the anchor to make that basin deeper or more accessible---converting trajectory-dependence into anchor-dependence.
\end{itemize}

\subsection{Steelmanning the Allele View}

Allele-based models are not wrong. Sierra et al.'s finding that cooperative species have lower cancer prevalence is real and important \cite{wagner2011}. Over evolutionary time, selection does modify anchors.

Our point is that the aggregate pattern ``cooperative = less cancer'' is equally consistent with:
\begin{enumerate}
    \item \textbf{Anchor modification:} Selection has reshaped the attractor landscape to favor high-repair basins.
    \item \textbf{Trajectory guidance:} Cooperative environments guide trajectories into high-repair attractor basins that already exist.
    \item \textbf{Genetic assimilation:} An initially trajectory-dependent response becomes anchor-encoded through sustained selection.
\end{enumerate}

These are non-identifiable from comparative data. Distinguishing them requires intervention experiments.

\subsection{When Nonergodicity Dominates}

\medskip
\noindent\fbox{\parbox{\dimexpr\linewidth-2\fboxsep-2\fboxrule}{%
\textbf{Taxonomy: When Trajectory Matters}

\smallskip
\textbf{Anchor-dominated (allele stories work):}
\begin{itemize}
    \item Low-dimensional development ($\Delta_D$ small)
    \item Few attractor basins
    \item Stable environments across generations
    \item Strong canalization
\end{itemize}

\textbf{Trajectory-dominated (nonergodicity matters):}
\begin{itemize}
    \item High-dimensional development ($\Delta_D \gg 0$)
    \item Many attractor basins accessible from same anchor
    \item Variable environments across generations
    \item Weak canalization, high plasticity
\end{itemize}
}}

\smallskip
Cancer risk, facial morphology, and human life-history traits sit squarely in the trajectory-dominated regime.

\subsection{Specific Empirical Predictions}

While the models often produce indistinguishable aggregate statistics, they diverge structurally. Table~\ref{tab:model_comparison} contrasts the mechanistic assumptions and specific predictions of each framework.

\begin{table}[ht]
    \centering
    \caption{\textbf{Anchor-based vs.\ Trajectory-based Models.} Indistinguishable from static data; diverge under intervention.}
    \label{tab:model_comparison}
    \renewcommand{\arraystretch}{1.3}
    \small
    \begin{tabular}{@{}p{0.22\textwidth} p{0.36\textwidth} p{0.36\textwidth}@{}}
        \toprule
        \textbf{Feature} & \textbf{Anchor Model (Allele-Based)} & \textbf{Trajectory Model (Nonergodic)} \\
        \midrule
        \textbf{Core Mechanism} &
        Genotype algorithmically determines phenotype. &
        Genotype anchors high-D system; trajectory determines which attractor. \\

        \textbf{Role of Environment} &
        \textit{Selection pressure:} Modulates which anchors are favored over generations. &
        \textit{Trajectory guidance:} Shapes which attractor basin is entered in real-time. \\

        \textbf{``Cooperative Safety''} &
        Selection modified anchors to favor low-cancer attractors. &
        Cooperative cues guide trajectories into existing low-cancer attractors. \\

        \midrule
        \multicolumn{3}{@{}l}{\textit{\textbf{Divergent Predictions (Testable)}}} \\
        \midrule

        \textbf{Environmental Shift} &
        \textbf{No change.} Trapped state is anchor-determined. &
        \textbf{Shift occurs.} New trajectory $\to$ new attractor. \\

        \textbf{Twin Worlds} &
        Identical anchors $\to$ identical phenotypes. &
        Identical anchors $\to$ divergent phenotypes ($F_{ST} \approx 0$, $P_{ST} \gg 0$). \\

        \textbf{GWAS} &
        ``Missing heritability'' = many small-effect variants. &
        ``Missing heritability'' = trajectory information projected out. \\
        \bottomrule
    \end{tabular}
\end{table}

\subsection{The Projection Problem}

Genotype-phenotype inference involves a cascade of projections:

\begin{center}
\begin{tabular}{lcl}
(anchor, trajectory, trapped state) & $\xrightarrow{\text{projection}}$ & (genotype, phenotype) \\
$(g, E_{0:T}, h_T) \in \R^{L+pT+m}$ & & $(g, x) \in \R^{L+n}$
\end{tabular}
\end{center}

Allele-based models further marginalize over trajectories:
\begin{equation}
    P(x \mid g) = \int P(x \mid g, E) P(E) \, dE
\end{equation}

This marginalization discards the trajectory information that determines which attractor basin was entered \cite{todd2025falsifiability,todd2025maxwell}. The ``genetic effect'' $\beta$ is not a property of the anchor alone but of the anchor-trajectory interaction averaged over some implicit distribution of environmental histories.

In our minimal model ($L=5$, $m=20$, $n=10$), $\Delta_D = 14$. In real organisms with $m \sim 10^4$ regulatory degrees of freedom, the dimensional gap is enormous---making trajectory-dependence correspondingly important.

\subsection{Connections to Related Work}

Our framework provides formal foundations for several recent proposals:

\textbf{Cancer memory} \cite{lissek2024}: Lissek argued that malignancy can progress through epigenetic mechanisms once initiated. In our terms: oncogenic changes perturb the developmental trajectory, pushing the system into a new attractor basin. The nonergodic dynamics \textit{are} the memory---the system cannot spontaneously return to its previous basin.

\textbf{Developmental plasticity} \cite{lee2022,letsou2024}: High-dimensional developmental systems with large $\Delta_D$ naturally exhibit plasticity---many attractor basins are accessible from a single anchor.

\textbf{Systems evolution} \cite{corning2022}: Evolution is primarily about functional organization, not individual genes. In our terms: selection modifies anchors, but anchors constrain organizational possibilities without fully determining them.

\textbf{Development-ageing-cancer} \cite{fontana2023}: Unified developmental models linking these processes are natural in a nonergodic framework, where trajectory history shapes multiple downstream outcomes.

In the tradition of critiques of algorithmic biology \cite{rosen1991}, we demonstrate that the gene-as-algorithm assumption is not merely an approximation but a projection that actively discards the trajectory structure of nonergodic development.


\section{Conclusion}

Biological development is nonergodic. The state space is too vast to explore; trajectories are trapped; phenotypes are attractor states, not algorithmic outputs.

The genome is a low-dimensional anchor that constrains which attractor basins exist, but environmental history determines which basin is entered. This anchor-trajectory duality resolves the genotype-phenotype puzzle: the mapping is many-to-one not because of noise or missing variants, but because the trajectory information that distinguishes outcomes is projected out by genotype-phenotype analysis.

Sierra et al.'s finding that cooperative species have lower cancer prevalence is real. Our contribution is to show that it admits both anchor-based (selection modified the genotype) and trajectory-based (environment guided development) interpretations---non-identifiable from comparative data, distinguishable only through intervention.

The ``missing heritability'' in GWAS is not missing variants. It is missing trajectories. The dimensional gap $\Delta_D$ quantifies how much information is lost when we project from the high-dimensional (anchor, trajectory, attractor) space to the low-dimensional (genotype, phenotype) summary.

Genotype does not algorithmically determine phenotype. Genotype anchors a nonergodic developmental system; environmental history traps it in an attractor; the phenotype is the trapped state. Recognizing this is essential for prediction, intervention, and understanding biological change.


\section*{Data and Code Availability}

Simulation code is available at \url{https://github.com/todd866/nonergodic-development}. The repository includes \texttt{developmental\_network.py}, which implements the developmental network (Equations 2--3, 5--7), generates all figures, and demonstrates the Twin Worlds experiment (Figure~\ref{fig:twins}).

\section*{Declaration of competing interest}
The author declares that there are no known competing financial interests or personal relationships that could have appeared to influence the work reported in this paper.

\section*{Declaration of generative AI and AI-assisted technologies in the writing process}
During the preparation of this work the author used Claude Code (Claude 4.5 Opus) for primary drafting and model development, with feedback from Gemini 3 Pro and GPT 5.1 Pro. After using these tools, the author reviewed and edited the content as needed and takes full responsibility for the content of the published article.


\bibliographystyle{unsrt}
\begin{thebibliography}{10}

\bibitem{sierra2025}
Sierra, C., et al. (2025). Coevolution of cooperative lifestyles and reduced cancer prevalence in mammals. \emph{Science Advances}, 11(46), eadw0685. \url{https://doi.org/10.1126/sciadv.adw0685}

\bibitem{singh2022}
Singh, M., \& Glowacki, L. (2022). Human social organization during the Late Pleistocene: Beyond the nomadic-egalitarian model. \emph{Evolution and Human Behavior}, 43(5), 418--431. \url{https://doi.org/10.1016/j.evolhumbehav.2022.07.003}

\bibitem{pigliucci2010}
Pigliucci, M. (2010). Genotype--phenotype mapping and the end of the `genes as blueprint' metaphor. \emph{Philosophical Transactions of the Royal Society B}, 365(1540), 557--566. \url{https://doi.org/10.1098/rstb.2009.0241}

\bibitem{alberch1991}
Alberch, P. (1991). From genes to phenotype: dynamical systems and evolvability. \emph{Genetica}, 84(1), 5--11. \url{https://doi.org/10.1007/BF00123979}

\bibitem{westeberhard2003}
West-Eberhard, M.J. (2003). \emph{Developmental Plasticity and Evolution}. Oxford University Press. \url{https://doi.org/10.1093/oso/9780195122343.001.0001}

\bibitem{todd2025falsifiability}
Todd, I. (2025). The limits of falsifiability: Dimensionality, measurement thresholds, and the sub-Landauer domain in biological systems. \emph{BioSystems}, 248, 105608. \url{https://doi.org/10.1016/j.biosystems.2025.105608}

\bibitem{todd2025maxwell}
Todd, I. (2025). Timing inaccessibility and the projection bound: Resolving Maxwell's demon for continuous biological substrates. \emph{BioSystems}, 249, 105632. \url{https://doi.org/10.1016/j.biosystems.2025.105632}

\bibitem{manolio2009}
Manolio, T.A., et al. (2009). Finding the missing heritability of complex diseases. \emph{Nature}, 461(7265), 747--753. \url{https://doi.org/10.1038/nature08494}

\bibitem{zuk2012}
Zuk, O., Hechter, E., Sunyaev, S.R., \& Lander, E.S. (2012). The mystery of missing heritability: Genetic interactions create phantom heritability. \emph{Proceedings of the National Academy of Sciences}, 109(4), 1193--1198. \url{https://doi.org/10.1073/pnas.1119675109}

\bibitem{waddington1957}
Waddington, C.H. (1957). \emph{The Strategy of the Genes}. Allen \& Unwin, London.

\bibitem{kauffman1993}
Kauffman, S.A. (1993). \emph{The Origins of Order: Self-Organization and Selection in Evolution}. Oxford University Press.

\bibitem{huang2009}
Huang, S. (2009). Non-genetic heterogeneity of cells in development: more than just noise. \emph{Development}, 136(23), 3853--3862. \url{https://doi.org/10.1242/dev.035139}

\bibitem{schlichting1998}
Schlichting, C.D., \& Pigliucci, M. (1998). \emph{Phenotypic Evolution: A Reaction Norm Perspective}. Sinauer Associates.

\bibitem{stearns1992}
Stearns, S.C. (1992). \emph{The Evolution of Life Histories}. Oxford University Press.

\bibitem{roff2002}
Roff, D.A. (2002). \emph{Life History Evolution}. Sinauer Associates.

\bibitem{aktipis2015}
Aktipis, C.A., et al. (2015). Cancer across the tree of life: cooperation and cheating in multicellularity. \emph{Philosophical Transactions of the Royal Society B}, 370(1673), 20140219. \url{https://doi.org/10.1098/rstb.2014.0219}

\bibitem{wagner2011}
Wagner, A. (2011). \emph{The Origins of Evolutionary Innovations: A Theory of Transformative Change in Living Systems}. Oxford University Press.

\bibitem{lissek2024}
Lissek, T. (2024). Cancer memory as a mechanism to establish malignancy. \emph{BioSystems}, 247, 105381. \url{https://doi.org/10.1016/j.biosystems.2024.105381}

\bibitem{lee2022}
Lee, C., Axtell, M., \& Bhattacharya, S. (2022). Evolution and maintenance of phenotypic plasticity. \emph{BioSystems}, 222, 104782. \url{https://doi.org/10.1016/j.biosystems.2022.104782}

\bibitem{letsou2024}
Letsou, W. (2024). The indispensable role of time in autonomous development. \emph{BioSystems}, 246, 105354. \url{https://doi.org/10.1016/j.biosystems.2024.105354}

\bibitem{corning2022}
Corning, P.A. (2022). A systems theory of biological evolution. \emph{BioSystems}, 214, 104630. \url{https://doi.org/10.1016/j.biosystems.2022.104630}

\bibitem{fontana2023}
Fontana, A. (2023). Unravelling the nexus: Towards a unified model of development, ageing, and cancer. \emph{BioSystems}, 234, 105066. \url{https://doi.org/10.1016/j.biosystems.2023.105066}

\bibitem{rosen1991}
Rosen, R. (1991). \emph{Life Itself: A Comprehensive Inquiry into the Nature, Origin, and Fabrication of Life}. Columbia University Press.

\end{thebibliography}

\end{document}
