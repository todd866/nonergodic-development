\documentclass[12pt]{article}

\usepackage[utf8]{inputenc}
\usepackage[T1]{fontenc}
\usepackage{lmodern}
\usepackage[margin=1in]{geometry}
\usepackage{amsmath,amssymb,amsthm}
\usepackage{graphicx}
\usepackage[numbers,square]{natbib}
\usepackage{hyperref}
\usepackage{booktabs}
\usepackage{setspace}
\doublespacing

% Theorems
\newtheorem{theorem}{Theorem}
\newtheorem{proposition}[theorem]{Proposition}
\newtheorem{lemma}[theorem]{Lemma}
\newtheorem{corollary}[theorem]{Corollary}
\theoremstyle{definition}
\newtheorem{definition}[theorem]{Definition}
\newtheorem{example}[theorem]{Example}
\theoremstyle{remark}
\newtheorem{remark}[theorem]{Remark}

% Shortcuts
\newcommand{\R}{\mathbb{R}}
\newcommand{\E}{\mathbb{E}}

\title{Genotype $\neq$ Phenotype: High-Dimensional Development, \\
Plasticity, and the Limits of Allele Stories}

\author{Ian Todd\\
Sydney Medical School\\
University of Sydney\\
\texttt{itod2305@uni.sydney.edu.au}}

\date{}

\begin{document}

\maketitle

\begin{abstract}
The assumption that genotype algorithmically determines phenotype underlies much of evolutionary modeling, including recent work linking cooperative lifestyles to reduced cancer prevalence. We prove that this framing is incomplete in a specific, quantifiable sense: when phenotype emerges from low-dimensional genetic parameters passing through a high-dimensional developmental system that integrates environmental information, allele-based and plasticity-based models become \textit{non-identifiable} from typical observational data. Using a minimal developmental network model, we demonstrate that (1) identical genotypes produce substantially different phenotypes---including different cancer mortality rates---under different environmental histories, (2) aggregate population patterns can arise equally well from either mechanism, and (3) these mechanisms diverge only in their predictions for environmental interventions. We formalize this via a ``dimensional gap'' $\Delta_D$ that quantifies when allele-centric models become uninformative. The allele story is not wrong but is a low-dimensional projection of a high-dimensional developmental reality.
\end{abstract}

\textbf{Keywords:} genotype-phenotype map, developmental plasticity, non-identifiability, dimensional collapse, cancer evolution, phenotypic plasticity

\section{Introduction}

A common implicit assumption in evolutionary biology is that genotype determines phenotype in an approximately algorithmic sense: genetic variants have ``effects'' on traits, and phenotypic evolution proceeds via changes in allele frequency. This framing underlies genome-wide association studies, quantitative genetics, and many theoretical models of trait evolution. Yet GWAS have famously failed to account for the majority of heritable variation in complex traits---the ``missing heritability'' problem \cite{manolio2009,zuk2012}---suggesting that simple additive models omit something fundamental about how genotypes produce phenotypes.

Recent work has extended this framework to cancer evolution. Sierra et al.\ (2025) demonstrated that cooperative mammalian species exhibit lower cancer prevalence than competitive, fast life-history species, and modeled this as selection on an ``oncogenic variant'' that increases late-life mortality \cite{sierra2025}. Their finding that cooperative environments select against cancer-promoting alleles while competitive environments can favor them (via a demographic Hydra effect) is compelling.

However, we argue that such allele-centric models are incomplete in a specific, quantifiable way. As Pigliucci \cite{pigliucci2010} emphasized, the ``genes as blueprint'' metaphor is ``not only woefully inadequate but positively misleading.'' The genotype does not algorithmically encode the phenotype; rather, genotype provides low-dimensional parameters to a high-dimensional developmental system that integrates environmental information over time \cite{alberch1991,westeberhard2003}. The phenotype---including cancer-relevant traits like somatic repair allocation---emerges from this process.

This matters because the same aggregate pattern (``cooperative species have less cancer'') can arise from two mechanistically distinct processes:
\begin{itemize}
    \item \textbf{Model A (allele-based):} Different alleles specify different late-life mortality rates $\mu_S$; selection favors lower $\mu_S$ in cooperative lineages.
    \item \textbf{Model B (plasticity-based):} Organisms possess flexible developmental policies that allocate resources between somatic repair and reproduction based on perceived environmental cues; cooperative environments induce higher repair allocation, yielding lower emergent $\mu_S$.
\end{itemize}

These models are \textit{non-identifiable} from aggregate cross-species data but diverge sharply in their predictions for interventions: Model A predicts that changing an organism's environment should not change its cancer risk (determined by alleles), while Model B predicts substantial phenotypic shifts in response to environmental change.

In this paper, we:
\begin{enumerate}
    \item Formalize a minimal developmental network model in which genotype parameterizes a dynamical system that maps environmental histories to phenotypes (Section~\ref{sec:model}).
    \item Prove a non-identifiability proposition showing that any allele-based pattern can be matched by a plasticity-based mechanism (Section~\ref{sec:proposition}).
    \item Demonstrate these results with simulations (Section~\ref{sec:results}).
    \item Discuss the broader implications for evolutionary interpretation and prediction (Section~\ref{sec:discussion}).
\end{enumerate}


\section{Background}

\subsection{The Gene-as-Algorithm Assumption}

Alberch \cite{alberch1991} introduced the concept of genotype--phenotype (G$\to$P) mapping to provide a framework for integrating genetics and developmental biology. The simplest such model treats the map $G \to P$ as approximately additive:
\begin{equation}
    P = \sum_i \beta_i G_i + \epsilon
\end{equation}
where $G_i$ are allelic states, $\beta_i$ are effect sizes, and $\epsilon$ is noise. This framing, while computationally tractable, obscures the developmental process that connects genotype to phenotype.

A more realistic view, dating to Waddington's ``epigenetic landscape'' \cite{waddington1957}, is that genotype provides parameters to a developmental dynamical system:
\begin{equation}
    \frac{dh}{dt} = f(h, e(t); g)
\end{equation}
where $h \in \R^m$ is a high-dimensional developmental state, $e(t)$ is environmental input, and $g$ is the genotype \cite{kauffman1993,huang2009}. The phenotype $x$ is then a readout of the final developmental state:
\begin{equation}
    x = \phi(h_T)
\end{equation}

In this view, the ``effect'' of a genetic variant is not a fixed quantity but depends on the environmental history the organism experiences---a phenomenon formalized in the reaction norm framework \cite{schlichting1998}.

\subsection{The Dimensional Gap}

We define a quantity that captures the mismatch between genetic and phenotypic dimensionality:

\begin{definition}[Dimensional Gap]
Let $L$ be the dimension of the genotype space, $m$ the dimension of the developmental state space, $n$ the dimension of the phenotype space, and $k$ the dimension of measured/observed traits. The \textbf{dimensional gap} is:
\begin{equation}
    \Delta_D = (m + n) - (L + k)
\end{equation}
\end{definition}

When $\Delta_D \gg 0$, the developmental system has far more degrees of freedom than the genetic input or measured output. This creates a many-to-one mapping at both ends: many developmental trajectories are consistent with any given genotype, and many mechanisms collapse to the same observed phenotype. (In our toy model, Section~\ref{sec:model}, $\Delta_D \approx 24$; in real organisms it will be much larger.)

Projection from high-dimensional to low-dimensional spaces destroys information in systematic ways \cite{todd2025falsifiability,todd2025maxwell}. Here we apply this insight to genotype-phenotype maps.

\subsection{Cooperative Lifestyles and Cancer}

Sierra et al.\ (2025) analyzed cancer prevalence across mammalian species and found a robust negative correlation between cooperative social structure and cancer mortality \cite{sierra2025}; see also \cite{aktipis2015} for broader context on cancer across the tree of life. Species with group living, plural breeding, and helper contributions exhibited lower cancer rates than solitary, competitive species.

Their theoretical model posits two genotypes:
\begin{itemize}
    \item G1: baseline late-life mortality $\mu_S$
    \item G2: oncogenic variant with mortality $\mu_S + \delta$
\end{itemize}

In competitive environments with high fertility, the G2 variant can be favored via a Hydra effect: removing older individuals frees resources for younger breeders. In cooperative environments where older individuals contribute as helpers, this effect reverses.

This is a compelling result. Our point is not that it is wrong, but that the interpretation---that cancer prevalence differences reflect allelic differences---is not uniquely supported by the data.

\subsection{Why Humans May Be a Special Case}

Singh and Glowacki (2022) argued that Late Pleistocene humans did not inhabit a single ``nomadic band'' niche but occupied a wide range of social ecologies: hierarchical and egalitarian, sedentary and mobile, small-scale and large-scale \cite{singh2022}. This ``diverse histories'' model implies that human ancestors were exposed to systematically variable environments across generations.

Under such conditions, selection should favor \textit{plastic developmental policies} over fixed allele-determined phenotypes \cite{westeberhard2003}. An organism that can adjust its repair-vs-reproduction allocation based on local environmental cues will outperform one locked into a single strategy.

This provides a principled reason to expect Model B (plasticity-based) mechanisms to be particularly important in humans.


\section{Model}\label{sec:model}

We define a minimal developmental network that captures the key features: low-dimensional genetic input, high-dimensional developmental dynamics, environmental modulation, and phenotypic readout.

\subsection{Developmental Dynamics}

Let $g \in \R^L$ be the genotype (with $L$ small), $e_t \in \R^p$ be the environmental input at time $t$ (with $p$ environmental dimensions), and $h_t \in \R^m$ be the developmental state (with $m \gg L$).

The developmental dynamics follow:
\begin{equation}\label{eq:dynamics}
    h_{t+1} = \sigma\left( W_h h_t + W_e e_t + W_g g \right)
\end{equation}
where $W_h, W_e, W_g$ are weight matrices and $\sigma$ is a nonlinear activation (we use $\tanh$).

\subsection{Phenotype and Cancer Risk}

The phenotype $x \in \R^n$ is a linear readout of the final developmental state:
\begin{equation}
    x = W_{out} h_T
\end{equation}

We extract a ``repair allocation'' variable $r \in [0,1]$ from the phenotype, representing the organism's investment in somatic maintenance versus reproduction/competition---a classic life-history tradeoff \cite{stearns1992,roff2002}. The emergent cancer mortality is:
\begin{equation}\label{eq:mu}
    \mu_S(x) = \mu_0 (1 - \alpha \cdot r(x))
\end{equation}
where $\mu_0$ is baseline mortality and $\alpha \in (0,1)$ is the efficacy of repair.

\subsection{Environmental Regimes}

We define two environmental regimes:
\begin{itemize}
    \item \textbf{Cooperative:} Low variance in resources, high social support, low conflict.
    \item \textbf{Competitive:} High variance in resources, low social support, high conflict.
\end{itemize}

\noindent\textbf{Parameter values:} Throughout, we use $L = 5$ (genotype dimension), $m = 20$ (developmental state dimension), $n = 10$ (phenotype dimension), and $p = 3$ (environmental input dimension). Cancer mortality parameters are $\mu_0 = 0.1$ (baseline) and $\alpha = 0.8$ (repair efficacy). The cooperative environment draws from $\mathcal{N}(0.5, 0.1)$ while the competitive environment draws from $\mathcal{N}(-0.5, 0.3)$.

\subsection{Causal Structure}

Figure~\ref{fig:dags} illustrates the causal difference between the two models:
\begin{itemize}
    \item \textbf{Model A:} $G \to \mu_S \to \text{Cancer}$, with environment only modulating selection on $G$.
    \item \textbf{Model B:} $G \to \text{Dev} \to \pi(E) \to r_t \to \mu_S \to \text{Cancer}$, with environment entering directly into development and policy.
\end{itemize}
Crucially, both graphs collapse to the same two-node summary (Environment $\to$ $\mu_S$) when projected onto low-dimensional observables---this is precisely the $\Delta_D$ problem.


\section{Non-Identifiability Result}\label{sec:proposition}

\begin{proposition}[Non-identifiability of allele vs.\ policy mechanisms]\label{prop1}
Let Model A be any allele-based system where late-life mortality takes values $\{\mu, \mu + \delta\}$ across two ecological regimes, with $0 \leq \mu < \mu + \delta \leq \mu_0$. Let Model B be any plasticity-based developmental system \textbf{capable of realizing any repair allocation $r \in [0,1]$ via some environmental history}, and where $\mu_S = \mu_0(1 - \alpha r)$ as in Equation~\ref{eq:mu}. Then for any pair $(\mu, \mu + \delta)$ satisfying these constraints, there exists an environment pair $(E_{\text{coop}}, E_{\text{comp}})$ such that the induced $\mu_S$ distributions in Model B match those in Model A for all species-level summary statistics.
\end{proposition}

\begin{proof}
Take any $(\mu, \mu + \delta)$ from Model A. We construct environments that induce matching $\mu_S$ distributions in Model B.

Choose $r_{\text{coop}}$ and $r_{\text{comp}}$ such that:
\begin{align}
    \mu_0(1 - \alpha r_{\text{coop}}) &= \mu \\
    \mu_0(1 - \alpha r_{\text{comp}}) &= \mu + \delta
\end{align}

Solving: $r_{\text{coop}} = (1 - \mu/\mu_0)/\alpha$ and $r_{\text{comp}} = (1 - (\mu + \delta)/\mu_0)/\alpha$.

For sufficiently expressive dynamics (e.g., the recurrent network of Equation~\ref{eq:dynamics}), there exists a mapping from environmental histories to any desired $r$; this is empirically illustrated in Figure~\ref{fig:same_genotype}D. Thus there exist environments $E_{\text{coop}}$ and $E_{\text{comp}}$ that induce these repair levels.

Therefore, any species-level summary (mean $\mu_S$, variance, etc.) that Model A produces can be exactly matched by Model B with appropriate environment choice. The models are observationally equivalent at this level.
\end{proof}

\begin{remark}
This result formalizes the sense in which allele models are ``incomplete'': they are not wrong, but they are \textit{non-identifiable} from aggregate data. The models become distinguishable only through interventions that change environment while holding genotype fixed.
\end{remark}


\section{Results}\label{sec:results}

\subsection{Same Genotype, Different Phenotypes}

Figure~\ref{fig:same_genotype} demonstrates that a single genotype produces substantially different phenotypes depending on environmental history.

\begin{figure}[htbp]
    \centering
    \includegraphics[width=\textwidth]{figures/fig1_same_genotype.pdf}
    \caption{Same genotype, different environments $\to$ different phenotypes. (A) Environmental histories for cooperative (blue) and competitive (red) regimes. (B) Developmental trajectories in state space (PCA projection). (C) Final phenotype vectors differ substantially. (D) Emergent repair allocation and cancer mortality: cooperative environments induce higher repair and lower $\mu_S$.}
    \label{fig:same_genotype}
\end{figure}

The key observation is panel (D): the same genotype produces different cancer mortality rates depending on developmental environment. This is the empirical foundation for Proposition~\ref{prop1}.

\subsection{Population-Level Patterns}

Figure~\ref{fig:population} shows that aggregate population patterns can arise from either allele-based or plasticity-based mechanisms.

\begin{figure}[htbp]
    \centering
    \includegraphics[width=\textwidth]{figures/fig2_population_patterns.pdf}
    \caption{Non-identifiability at the population level. (A) Model B (plastic policy): same genetic distribution in different environments produces different $\mu_S$ distributions. (B) Model A (allele-based): different genetic distributions can produce similar patterns. (C) Aggregate statistics (mean $\pm$ SD of $\mu_S$) are indistinguishable between models.}
    \label{fig:population}
\end{figure}

\subsection{Causal Structure}

Figure~\ref{fig:dags} shows the causal graphs for both models. The key difference is that environment is purely a selection context in Model A but a direct developmental input in Model B; this distinction underlies the non-identifiability result. Critically, both causal structures collapse to the same two-node summary (Environment $\to$ $\mu_S$) when projected onto low-dimensional observables---this is precisely where $\Delta_D > 0$ creates ambiguity.

\begin{figure}[htbp]
    \centering
    \includegraphics[width=\textwidth]{figures/fig5_causal_dags.pdf}
    \caption{Causal structure of the two models. (A) Model A: genotype directly determines $\mu_S$; environment only modulates selection. (B) Model B: genotype parameterizes a developmental system; environment enters directly into development and policy, producing emergent $\mu_S$.}
    \label{fig:dags}
\end{figure}

\subsection{Information Loss Under Projection}

Figure~\ref{fig:projection} illustrates the information loss that results from dimensional projection. Panel C quantifies this: in the full model, environment explains most of the variance in $\mu_S$, but an allele-only regression would misattribute much of this variance to ``genetic effects.''

\begin{figure}[htbp]
    \centering
    \includegraphics[width=\textwidth]{figures/fig3_projection_loss.pdf}
    \caption{Information loss under projection. (A) In full genotype $\times$ environment space, cooperative and competitive regimes are clearly separated. (B) In projected (allele-only) space, environment is hidden. (C) Variance decomposition: environment explains most variance in $\mu_S$, but allele models attribute this to genetic effects.}
    \label{fig:projection}
\end{figure}

\subsection{Twin Worlds Experiment}

Figure~\ref{fig:twins} provides a decisive demonstration: identical genotype distributions in different worlds produce patterns that naive allele models would interpret as genetic differences. A naive comparative study could infer different ``oncogenic allele frequencies'' between worlds that in fact share identical genotype distributions.

\begin{figure}[htbp]
    \centering
    \includegraphics[width=\textwidth]{figures/fig6_twin_worlds.pdf}
    \caption{Twin worlds experiment. (A) Same genotype distribution developed in cooperative vs.\ competitive worlds produces dramatically different $\mu_S$ distributions. (B) A naive allele model would infer different ``oncogenic allele frequencies'' despite no genetic difference. (C) Genotype has minimal correlation with $\mu_S$; environment determines the outcome.}
    \label{fig:twins}
\end{figure}

\subsection{Divergent Predictions}

Figure~\ref{fig:prediction} shows where the models make different predictions. This makes environmental interventions the critical empirical test for discriminating allele- vs.\ policy-based mechanisms.

\begin{figure}[htbp]
    \centering
    \includegraphics[width=\textwidth]{figures/fig4_prediction_divergence.pdf}
    \caption{Divergent predictions for environmental change. (A) Model A predicts no change in $\mu_S$ when environment shifts; Model B predicts substantial reduction. (B) Individual trajectories show systematic phenotypic change without genetic change.}
    \label{fig:prediction}
\end{figure}


\section{Discussion}\label{sec:discussion}

\subsection{Steelmanning the Allele View}

We emphasize that allele-based models are not wrong. Sierra et al.'s finding that cooperative species have lower cancer prevalence is real and important. Selection does act on heritable variation, and over evolutionary time, lineages in cooperative environments likely do accumulate genetic changes that support higher somatic investment \cite{wagner2011}.

Our point is narrower: the aggregate pattern ``cooperative = less cancer'' is equally consistent with:
\begin{enumerate}
    \item Selection having already fixed cancer-suppressing alleles in cooperative lineages (Model A)
    \item Selection having favored plastic developmental policies that respond to cooperative cues (Model B)
    \item Both (which is probably the truth)
\end{enumerate}

The models are non-identifiable from cross-species comparative data. Distinguishing them requires environmental manipulation experiments.

\subsection{When Allele Stories Work vs.\ Fail}

\medskip
\noindent\fbox{\parbox{\dimexpr\linewidth-2\fboxsep-2\fboxrule}{%
\textbf{Taxonomy: When Allele Stories Become Non-Identifiable}

\smallskip
\textbf{Allele stories work when:}
\begin{itemize}
    \item Traits are low-dimensional
    \item Plasticity is weak
    \item Environments are stable
    \item $\Delta_D$ is small
\end{itemize}

\textbf{Allele stories become non-identifiable when:}
\begin{itemize}
    \item $\Delta_D \gg 0$ (high-dimensional development)
    \item Environments vary across generations
    \item Phenotype is policy-based rather than algorithm-generated
    \item Identical genotypes produce divergent phenotypes depending on $E(t)$
    \item Low-dimensional measurement destroys mechanistic information
\end{itemize}
}}

Cancer risk in cooperative vs.\ competitive mammals, facial morphology, and human life-history traits sit squarely in the second category.

\subsection{Specific Empirical Predictions}

The models make divergent predictions that are in principle testable:

\textbf{Prediction 1:} Competitive-raised individuals from cooperative species should temporarily exhibit elevated cancer risk (higher $\mu_S$) even with no genetic change.

\textbf{Prediction 2:} Cooperative-raised individuals from competitive species should show increased repair allocation (higher $r$) and lower emergent $\mu_S$.

\textbf{Prediction 3:} GWAS-style \textit{comparative} allele scanning for cancer across species will detect ``genetic effects'' that diminish or vanish once developmental or environmental covariates are included.

\subsection{Dimensional Constraints on Interpretation}

The key insight is that genotype-phenotype maps involve multiple projections:

\begin{center}
\begin{tabular}{lcl}
High-D developmental state & $\xrightarrow{\text{projection}}$ & Low-D observed phenotype \\
$(g, E, h_T) \in \R^{L+pT+m}$ & & $x \in \R^n$
\end{tabular}
\end{center}

Allele-based models further project by marginalizing over environment:
\begin{equation}
    P(x \mid g) = \int P(x \mid g, E) P(E) \, dE
\end{equation}

This marginalization loses information about the environment-dependent structure \cite{todd2025falsifiability,todd2025maxwell}. The ``genetic effect'' $\beta$ that emerges is not a property of the allele alone but of the allele-environment interaction averaged over some implicit environmental distribution.

Even in our minimal toy model, with $L=5$ genetic dimensions, $m=20$ developmental state dimensions, $n=10$ phenotype dimensions, and $k=1$ measured trait ($\mu_S$), we have $\Delta_D = (20+10)-(5+1) = 24$. In real organisms, $\Delta_D$ will be vastly larger, making non-identifiability correspondingly worse.


\section{Conclusion}

The allele-centric view is a special case of a broader, dimensionality-constrained developmental mapping. When $\Delta_D$ is large---when development is high-dimensional and plastic---allele effects become non-identifiable from observational data, and plastic policies dominate phenotype formation.

Cancer prevalence in cooperative vs.\ competitive species is an instance of this general phenomenon. The pattern Sierra et al.\ documented is real; our contribution is to show that it admits multiple mechanistic interpretations that are indistinguishable from aggregate data but make different predictions for intervention.

Genotype does not algorithmically determine phenotype. Genotype provides parameters to a high-dimensional developmental system that integrates environmental information to produce phenotypic outcomes. Recognizing this is essential for prediction, intervention, and understanding rapid phenotypic change.


\section*{Data and Code Availability}

Simulation code is available at \url{https://github.com/todd866/genotype-vs-phenotype}. The repository includes \texttt{developmental\_network.py}, which implements the model (Equations 1--3), generates all figures, and demonstrates the Twin Worlds experiment (Figure~\ref{fig:twins}).

\section*{Declaration of competing interest}
The author declares that there are no known competing financial interests or personal relationships that could have appeared to influence the work reported in this paper.

\section*{Declaration of generative AI and AI-assisted technologies in the writing process}
During the preparation of this work the author used Claude Code (Claude 4.5 Opus) for primary drafting and model development, with feedback from Gemini 3 Pro and GPT 5.1 Pro. After using these tools, the author reviewed and edited the content as needed and takes full responsibility for the content of the published article.


\bibliographystyle{unsrt}
\begin{thebibliography}{10}

\bibitem{sierra2025}
Sierra, C., et al. (2025). Coevolution of cooperative lifestyles and reduced cancer prevalence in mammals. \emph{Science Advances}, 11(46), eadw0685. \url{https://doi.org/10.1126/sciadv.adw0685}

\bibitem{singh2022}
Singh, M., \& Glowacki, L. (2022). Human social organization during the Late Pleistocene: Beyond the nomadic-egalitarian model. \emph{Evolution and Human Behavior}, 43(5), 418--431. \url{https://doi.org/10.1016/j.evolhumbehav.2022.07.003}

\bibitem{pigliucci2010}
Pigliucci, M. (2010). Genotype--phenotype mapping and the end of the `genes as blueprint' metaphor. \emph{Philosophical Transactions of the Royal Society B}, 365(1540), 557--566. \url{https://doi.org/10.1098/rstb.2009.0241}

\bibitem{alberch1991}
Alberch, P. (1991). From genes to phenotype: dynamical systems and evolvability. \emph{Genetica}, 84(1), 5--11. \url{https://doi.org/10.1007/BF00123979}

\bibitem{westeberhard2003}
West-Eberhard, M.J. (2003). \emph{Developmental Plasticity and Evolution}. Oxford University Press. \url{https://doi.org/10.1093/oso/9780195122343.001.0001}

\bibitem{todd2025falsifiability}
Todd, I. (2025). The limits of falsifiability: Dimensionality, measurement thresholds, and the sub-Landauer domain in biological systems. \emph{BioSystems}, 248, 105608. \url{https://doi.org/10.1016/j.biosystems.2025.105608}

\bibitem{todd2025maxwell}
Todd, I. (2025). Timing inaccessibility and the projection bound: Resolving Maxwell's demon for continuous biological substrates. \emph{BioSystems}, 249, 105632. \url{https://doi.org/10.1016/j.biosystems.2025.105632}

\bibitem{manolio2009}
Manolio, T.A., et al. (2009). Finding the missing heritability of complex diseases. \emph{Nature}, 461(7265), 747--753. \url{https://doi.org/10.1038/nature08494}

\bibitem{zuk2012}
Zuk, O., Hechter, E., Sunyaev, S.R., \& Lander, E.S. (2012). The mystery of missing heritability: Genetic interactions create phantom heritability. \emph{Proceedings of the National Academy of Sciences}, 109(4), 1193--1198. \url{https://doi.org/10.1073/pnas.1119675109}

\bibitem{waddington1957}
Waddington, C.H. (1957). \emph{The Strategy of the Genes}. Allen \& Unwin, London.

\bibitem{kauffman1993}
Kauffman, S.A. (1993). \emph{The Origins of Order: Self-Organization and Selection in Evolution}. Oxford University Press.

\bibitem{huang2009}
Huang, S. (2009). Non-genetic heterogeneity of cells in development: more than just noise. \emph{Development}, 136(23), 3853--3862. \url{https://doi.org/10.1242/dev.035139}

\bibitem{schlichting1998}
Schlichting, C.D., \& Pigliucci, M. (1998). \emph{Phenotypic Evolution: A Reaction Norm Perspective}. Sinauer Associates.

\bibitem{stearns1992}
Stearns, S.C. (1992). \emph{The Evolution of Life Histories}. Oxford University Press.

\bibitem{roff2002}
Roff, D.A. (2002). \emph{Life History Evolution}. Sinauer Associates.

\bibitem{aktipis2015}
Aktipis, C.A., et al. (2015). Cancer across the tree of life: cooperation and cheating in multicellularity. \emph{Philosophical Transactions of the Royal Society B}, 370(1673), 20140219. \url{https://doi.org/10.1098/rstb.2014.0219}

\bibitem{wagner2011}
Wagner, A. (2011). \emph{The Origins of Evolutionary Innovations: A Theory of Transformative Change in Living Systems}. Oxford University Press.

\end{thebibliography}

\end{document}
